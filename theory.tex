# Theory

We want to visualize an arbitrary \(n\)-dimensional Gaussian distribution,
  \(N(\mu, \Sigma)\),
with mean vector
  \(\mu \equiv \{\mu_1, \cdots, \mu_n\}\)
and covariance matrix \(\Sigma\).

A single draw from \(N(\cdot)\) requires \(n\) draws from the standard
1-dimensional normal distribution \cite{3569}
(or, equivalently, a draw from the standard \(n\)-dimensional normal
\cite{hennig-tr-is-8}), as follows.
Let
  \(\epsilon \equiv \{\epsilon_1, \cdots, \epsilon_n\}\)
be \(n\) i.i.d. Gaussian values.
Let
  \(L \equiv \text{Chol}(\Sigma)\)
be the (lower-triangular) Cholesky decomposition of the covariance matrix
\(\Sigma\).
Then the vector
  \begin{equation}
    gamma \equiv \mu + L \epsilon
    \label{eqn:oneRandomDraw}
  \end{equation}
is a single draw from \(N(\cdot)\).

The goal of animation is to produce multiple draws, \(\gamma(t)\), whose
marginal distribution is always equal to \(N(\cdot)\):
  \(\gamma(t) ~ N(\mu, \Sigma) \,\, \forall t\).
Thanks to Equation \ref{eqn:oneRandomDraw}, it is sufficient to construct an
\(\epsilon(t)\) which is marginally drawn from the _standard_ normal:
  \(\epsilon(t) ~ N(0, I) \,\, \forall t\)
\cite{hennig-tr-is-8}.

\cite{hennig-tr-is-8} treated \(\epsilon\) holistically, and exploited the
spherical symmetry of the distribution.
This paper instead treats the \(\{\epsilon_i\}(t)\) as independent copies of
a single type of object: the _standard Gaussian oscillator_.
The present approach yields paths with more variety, and is simpler both in
theory and implementation.

## Gaussian oscillators

A standard Gaussian oscillator is a set of random variables, \(\{X_t\}\),
indexed by the continuous variable \(t\), with the following properties:

1. Marginally, \(X_t ~ N(0, 1) \forall t\).
2. \(\lim_{\Delta t \rightarrow 0} \langle X_t X_{t + \Delta t} \rangle = 0
   \forall t\)

Informally, one can imagine the oscillator as a particle moving back and forth
along the real line as time, \(t\), passes.  The expectation of the fraction of
time it spends at each position \(x\) is \(N(x; 0, 1)\).

Gaussian oscillators have usually been constructed by interpolation.
First, the integer-valued times
evenly-spaced values of \(t\).  Then

## Oscillator populations versus individual oscillators
